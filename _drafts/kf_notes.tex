---
title: Kf_notes
date: 2018-09-12 12:01:08.261000000 -07:00
---

\documentclass[10pt]{article}
\usepackage{amsmath}
\usepackage{amsfonts}
\usepackage{amssymb}
\begin{document}

This explanation flips the script on normal introductions to the Kalman Filter (KF), working working backwards from a full particle filter (PF). This approach makes much more sense because PFs make much more sense - simulate a lot of possibilities, pick the best one, and continue.

Fun application: PFs can be used in multi-modal filters for GPS.
Going from a PF to a KF entails adding stronger assumptions about the system:
\begin{itemize}
\item[PF $\Longrightarrow$ Unscented KF (UKF)]: assume gaussian distribution requires unimodal distribution.
\item[UKF $\Longrightarrow$ EKF]: assume mean propagates, differentiable dynamics, find Jacobians.
\item[EKF $\Longrightarrow$ KF]: Linear system
\end{itemize}


$\hat{x}$ - estimate
$$z = h(x) + v$$
$$x_{k+1} = f(x_k,t) + v$$
$$\hat{x}_{k+1} = f(\hat{x}_k,t)$$

$$P_k = E((x_k-\hat{x}_k)(x_k-\hat{x}_k)^T)$$
Diagonal matrix of variances if we have uncorrelated noise

$$\hat{x}_{k+1} + \Delta x_{k+1} = f(\hat{x}_k) + \frac{\delta f}{dx} |_{\hat{x}_k} \Delta x_k $$
$$\hat{x}_{k+1} = F\Delta x_k + v$$
F is derivative of the dynamics at the state $\hat{x}_k$
Solve for $\Delta x \Delta x ^T$
$$E(F\Delta x \Delta x^T F^T + v v^T + F\Delta x_k v^T + v \Delta x^T F^T)$$
Expected value of last two terms is zero if noise is white and uncorrelated 

$$P_{k+1} = E(F \Delta x_k Delta x_k^T F^T) + E (v v^T)$$
$$P_{k+1} = FP_kF + Q$$

Run extended kalman filter
$$\hat{x}_{k+1}^* = f(\hat{x}_k)$$
$$F = \frac{\delta}{dx} = \texttt{make}F(\hat{x})$$
$$P^*_{k+1} = FP_kF^T + Q$$
$$\hat{z}_{k+1} = h(\hat{x}_{k+1})$$
$$y = z_{k+1} - \hat{z}_{k+1}$$
Want to know covariance of y (this will be S)
$$S = HP*H^T + R$$
$H$ is linearized h 
$K$ - kalman gain
$$K = P*H^TS^{-1}$$
$$\hat{x}_{k+1} = \hat{x}*_{k+1} + Ky$$
$$P_{k+1} = E((\hat{x}_{k+1}-x_{k+1})(\hat{x}_{k+1}-x_{k+1})^T)$$
$$P_{k+1} =  P*_{k+1}- KHP*_{k+1}$$
Process
\begin{enumerate}
\item Figure out $f$ and $h$
\item Figure out $F$ and $H$
\item Figure out $R$ and $Q$
\item Initial state and uncertainty
\end{enumerate}

Where things fall apart:
Negative eigenvalues for covariance matricies can happen $P*$ becomes negative 
How to avoid this: Joseph Form.
$$jf = (1-KH)P*(1-KH)^T+KRK^T$$

What if you take $P$ and instead use $P = UDU^T$ makes numerical problems go away UD filters 32 bit or less

What if you have unobservable errors? Considerate covariance - add a bit to covariance $P*_{k+1} = P*_{k+1} + F$ Need to use joeseph form to do updates.

Speed - 
Big problem is the matrix inverse in $K = P*H^TS^{-1}$
if $R$ is a diagonal matrix, do a bunch of scalar equations 

How do you know your filter is running properly 
Two covariance matrices 
Record real error over time vs calculated covariance. See if distribution of real error fits inside calculated covariance. 
Real systems: y should be uncorrelated over time 

Real filters
IMU type filters 
$$x_{k+1} = Fx_{k} + Gu_k$$
$$x_{k+1} = f(x_k,t,u_k)$$

Don't model real noise because you have a very good IMU 
For an IMU type filter, just use $\omega$ as the Input
$z = [q \theta \gamma \beta]^T$
Process noise becomes ability to sense

Q: best way to deal with kalman filtering 6dof+ systems?
Q: how do you come up with the probability distribution ?  
Q: sensor noise independence?
 \end{document}
